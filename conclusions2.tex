\chapter{Conclusions}

\emph{In this last chapter we conclude our work enumerating the main conclusions. Finally, we propose future work which can be performed on the topic of this Master Thesis.}

\section{Conclusions}
In this Thesis, we have examined central aspects of Personal Cloud storages 
services to characterize their performance in two different ways, data transfers
and synchronization protocol.

First of all, we present a measurement study of various Personal Clouds: Dropbox, Box and SugarSync.
During the measurement we employed two different scenarios to execute our tests:
University laboratories and PlanetLab. With the use of these two platforms we could
evaluate how the geographic location of a client impacts on the QoS. Furthermore, we analyze
daily patterns in services and transfer speeds for the services.

From this work we conclude:
\begin{itemize}
	\item The transfer performance of these services 
	\textit{greatly varies from one provider to another}, which is a valuable 
	piece of information for designers and developers.
	
	\item The \textit{geographic location} of a client importantly impacts on
	the speed of transfers. For instance, North American clients experience 
	transfers several times faster than European ones for the same Personal Cloud. 
	
	\item The \textit{variability} of transfers depends on several
	factors, such as the \textit{traffic type (in/out) or the hour of the day}.
	Actually, we found daily patterns in the DropBox service.
\end{itemize}

After the measurement, we study the sync protocol of StackSync, our open source Personal Cloud,
and other private solutions. To perform this tests we developed a novel trace generator
that could be useful for other researchers to simulate user behaviour. We have also extended
a benchmarking tool to achieve realistic tests with our generated traces.

As a result of these experiments, we can conclude that StackSync is going in the right
way as a Personal Cloud. In the comparison with other private solutions, we have demonstrated that,
in terms of overhead, we have done a good work.

\section{Future Work}
There are some improvements that can be done to this Thesis:

\begin{itemize}
	\item \textbf{CPU and RAM monitorization}. The benchmarking tool can be improved by monitoring
	the usage of CPU and RAM. With this improve we will understand how much computation power
	needs a desktop client to process user actions.
	\item \textbf{Implement new benchmarks}. In this work we created an additional benchmark that captures
	the traffic overhead. In future works it is possible to implement other types of benchmarks, for example,
	synchronize a log file.
	\item \textbf{Create realistic files}. With our trace generator we can simulate user behaviour, but
	we only synchronize binary random files. An improvement could be to generate files with real content
	and with different file extensions, such as .docx, .xml or .mp3.
\end{itemize}

Furthermore, the results exposed in this thesis could be used to implement new improvements in the StackSync
client. For example, updating files generates a huge overhead. Now that we know this issue, we can work
on advanced mechanisms to reduce overhead when synchronizing updates.
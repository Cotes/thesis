\chapter{Conclusions}

In this article, we have introduced StackSync, an open framework for Personal Cloud systems.
Its architecture is highly modular, with each module represented by a well-defined API,
allowing researchers to replace components for innovation in versioning,
deduplication, live synchronization or continuous reconciliation, among other relevant topics. 
StackSync provides a reference implementation and useful tools for rapid prototyping and evaluation. 
The reference implementation of the file synchronization engine has been built
on top of a lightweight MOM-RPC middleware, called ObjectMQ, whose one-to-one and one-to-many
abstractions has considerably simplified the design of StackSync. This middleware is ideally
suited to support push notification over persistent connections,
which is critical for live synchronization.  

StackSync is now under active development in the context of the FP7 CloudSpaces project
with the collaboration of several partners. Further work includes adding privacy measures and
supporting adaptive storage in collaboration with EPFL and EURECOM. 
Since today most of these systems are proprietary, relying on infrastructures that are
invisible to the research community, we believe that StackSync will help to advance state
of the art in Personal Cloud systems.
\chapter{Introduction}

In the last few years, we have witnessed a clear shift from Personal Computers and local storage to multi-device access and cloud storage services. The Personal Computer (PC) paradigm is slowly waning as traditional PCs are being outnumbered by other devices like smartphones and tablets. This means that in the next years  users will access their digital lives in the cloud from a variety of heterogeneous devices and platforms. This is clearly reflected by the massive adoption of services like Dropbox, Google Drive or OneDrive, among others.  The term ``Personal Cloud'' has been coined in the last years to reflect this trend and many analysts forecast  a bright future for these solutions~\cite{forrester}\cite{gartner}. To clarify the term we propose here the so-called 3S Personal Cloud Definition:

\textit{The Personal Cloud is a unified digital  locker for our personal data offering three key services: Storage, Synchronization and Sharing.  On the one hand, it must provide redundant and trustworthy cloud data storage for our information flows irrespective of their type. On the other hand, it must provide syncing and file exploring capabilities in different heterogeneous platforms and devices. And finally, it must offer fine-grained information sharing to third-parties (users and applications).}

Despite the much trumpeted success of the so-called cloud storage revolution, exemplified by Dropbox and the 
likes, little is known about the architecture of these commercial solutions, and existing open source alternatives 
(e.g., ownCloud\footnote{\url{ http://owncloud.org/ }}, SparkleShare\footnote{\url{ http://sparkleshare.org/}})
fall short of addressing all the requirements of the Personal Cloud, mainly in terms of scalability and extensibility. 
In this sense, architecturing a Personal Cloud is not trivial, as this model is still in early stages, and no reference architecture exists.

Architectural complexity is well captured by file synchronization, as the success of Personal Cloud services
lies in great measure in the scalability of their sync services and sync protocol. In this cloud paradigm, a desktop client
software typically keeps all the files in a specified folder in sync with the servers, automatically synchronizing changes across the devices of the same user. Since files can be shared with other users, changes in
shared folders must be also synced with every account that has been given access to the shared folders.

AQUI DECIR ALGO DE LAS TRAZAS Y DEL BENCHMARKING TOOL

\section{Problem statement}
...
\begin{itemize}
	\item \textbf{No open architecture for Personal Clouds}. Commercial Personal Clouds solutions like Dropbox
	are propietary, and rely on algorithms that are \textit{invisible} to the research community and existing open source alternatives fall short of addressing all the requirements of the Personal Clouds.
	\item \textbf{Lack of real traces}. Although there are some articles that study user behavior in Personal Clouds,
	there are not real traces available for the research community.
	\item \textbf{Simple benchmarking tools}. Current benchmarking tools for Personal Clouds are simple and not enough to
	analize deeply their performance.
\end{itemize}

\section{Contributions of this Thesis}
We propose \texttt{StackSync}: an open source Personal Cloud. \texttt{StackSync} is, as far as we know, the first
open source Personal Cloud that satisfies all the requirements from the 3S Personal Cloud definition. In this work we put more emphasis on the sync protocol, which is the core service of any Personal Cloud, and on the validation in order to compare our novel implementation with other commercial solutions.

Our major contributions are the following:

\begin{itemize}
	\item \textbf{Analysis of the state-of-the-art}. Prior to designing our solution, we study the current approaches concerning Personal Cloud systems which is reported further in this manuscript.
	\item \textbf{Design and implementation of sharing capabilities for StackSync}. StackSync lacked sharing capabilites to accomplish the requirements from the 3S definition. 
	\item \textbf{Generation of realistic traces}. To evaluate the system, we developed a benchmarking tool to generate
realistic workloads. We implemented this tool because we found no publicly available trace containing both 
the files and the history of modifications to those files that allowed us to evaluate our file-syncing service.
	\item \textbf{Evaluation and validation of StackSync}. StackSync has been extensively tested using generated traces to validate the syncing protocol and the efficient use of resources. Furthermore, we extended an open benchmark \cite{drago2013benchmarking} and we compared our service with Dropbox, Box, OneDrive, and Google Drive.
\end{itemize}

\section{Thesis structure}
After this brief introduction, Chapter 2 introduces the related work and background needed to understand concepts
and definitions that we will use throughout this thesis. Among others, it discusses existing Personal Cloud solutions and advanced features that a Personal Cloud should implement and presents the architecture of StackSync.

Chapter 3 focuses on the design and implementation of the traces generator, it is explained in detail how we implement sharing in StackSync and how we modified the benchmarking tool developed by Drago et al.~\cite{drago2013benchmarking} to
conduct trace-driven experiments with the output of our generator.

In Chapter 4 we evaluate StackSync and compare it with other Personal Clouds. We conclude this thesis in Chapter 5, where we draw some conclusions and enumare the achieved goals throught our study. Finally, we propose future work which can be performed on the topic of this Master Thesis.

\section{Publications}

lalalalal

lalalala